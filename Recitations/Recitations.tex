\documentclass[fleqn, a4paper, 12pt, twoside]{article}

\newcounter{recitationcount} %creates a new counter for recitation numbers (must be executed before exsheets is loaded)
\newcommand\recitation{\refstepcounter{recitationcount}}

\usepackage[counter-within = recitationcount]{exsheets}
\usepackage{amsmath, amssymb, amsthm} %standard AMS packages
\usepackage{marginnote} %marginnotes
\usepackage{gensymb} %miscellaneous symbols
\usepackage{commath} %differential symbols
\usepackage{xcolor} %colours
\usepackage{cancel} %cancelling terms
\usepackage{siunitx} %formatting units
\usepackage{tikz, pgfplots} %diagrams
	\usetikzlibrary{calc, hobby, patterns, intersections}
\usepackage{graphicx} %inserting graphics
\usepackage{hyperref} %hyperlinks
\usepackage{datetime} %date and time
\usepackage{ulem} %underline for \emph{}
\usepackage{xfrac, lmodern} %inline fractions
\usepackage{enumerate} %numbered lists
\usepackage{float} %inserting floats
\usepackage{circuitikz} %circuit diagrams

\newcommand\numberthis{\addtocounter{equation}{1}\tag{\theequation}} %adds numbers to specific equations in non-numbered list of equations

\newcommand{\AxisRotator}[1][rotate=0]{
	\tikz [x=0.25cm,y=0.60cm,line width=.2ex,-stealth,#1] \draw (0,0) arc (-150:150:1 and 1);%
} %rotation symbols on axes

\theoremstyle{definition}
\newtheorem{example}{Example}
\newtheorem{definition}{Definition}

\theoremstyle{theorem}
\newtheorem{theorem}{Theorem}

\newcommand{\curl}{\mathrm{curl\,}}

\makeatletter
\@addtoreset{section}{part} %resets section numbers in new part
\makeatother

\newcommand\blfootnote[1]{%
	\begingroup
	\renewcommand\thefootnote{}\footnote{#1}%
	\addtocounter{footnote}{-1}%
	\endgroup
}

\RenewQuSolPair{question}[name=Recitation \therecitationcount\ -- Exercise]{solution}[name=Recitation \therecitationcount\ -- Solution]

\SetupExSheets{solution/print = true, totoc = true} %prints all solutions by default

%opening
\title{Differential and Integral Calculus : Recitations}
\author{Aakash Jog}
\date{2014-15}

\begin{document}

\maketitle
%\setlength{\mathindent}{0pt}

\blfootnote
{	
	\begin{figure}[H]
		\includegraphics[height = 12pt]{cc.eps}
		\includegraphics[height = 12pt]{by.eps}
		\includegraphics[height = 12pt]{nc.eps}
		\includegraphics[height = 12pt]{sa.eps}
	\end{figure}
	This work is licensed under the Creative Commons Attribution-NonCommercial-ShareAlike 4.0 International License. To view a copy of this license, visit \url{http://creativecommons.org/licenses/by-nc-sa/4.0/}.
} %CC-BY-NC-SA licencse

\tableofcontents

\newpage
\section{Instructor Information}

\textbf{Michael Bromberg}\\
~\\
E-mail: \href{mailto:micbromberg@gmail.com}{micbromberg@gmail.com}\\

\newpage

\part{Sequences and Series}

\section{Sequences}

\recitation

\begin{question}
	Prove:
	\begin{equation*}
		\lim\limits_{n \to \infty} \dfrac{2n^2 + n + 1}{n^2 + 3} = 2
	\end{equation*}
\end{question}

\begin{solution}[print]
	Let
	\begin{equation*}
		\varepsilon > 0
	\end{equation*}

	\begin{align*}
		\left| \dfrac{2n^2 + n + 1}{n^2 + 3} - 2 \right| &= \left| \dfrac{2n^2 + n + 1 - 2n^2 - 6}{n^2 + 3} \right|\\
		&= \left| \dfrac{n - 5}{n^2 + 3} \right| \\
		&\leq \left| \dfrac{n - 5}{n^2} \right|\\
		&\leq \dfrac{1}{n}\\
		&< \varepsilon
	\end{align*}
	Therefore, let $N = \left[ \dfrac{1}{\varepsilon} \right] + 1$.
	Hence, for this $N$, $|a_n - L| < \varepsilon$.\\
	Therefore, $\lim\limits_{n \to \infty} \dfrac{2n^2 + n + 1}{n^2 + 3} = 2$.
	\qed
\end{solution}

\begin{question}
	Prove
	\begin{equation*}
		\lim\limits_{n \to \infty} \dfrac{n^3 + \sin n + n}{2n^4} = 0
	\end{equation*}
\end{question}

\begin{solution}[print]
	Let $\varepsilon > 0$
	\begin{align*}
		\left| \dfrac{n^3 + \sin n + n}{2n^4} \right| &\leq \left| \dfrac{n^3 + 1 + n}{2n ^4} \right|\\
		&\leq \left| \dfrac{3n^3}{2n^4} \right| = \dfrac{3}{2} \cdot \dfrac{1}{n} < \varepsilon
	\end{align*}
	Therefore, let $N = \left[ \dfrac{3}{2 \varepsilon} \right] + 1$.
	Hence, for this $N$, $|a_n - L| < \varepsilon$.\\
	Therefore, $\lim\limits_{n \to \infty} \dfrac{n^3 + \sin n + n}{2n^4} = 0$
	\qed
\end{solution}

\begin{question}
	Calculate $\sqrt[3]{n^3 + 3n} - n$.
\end{question}

\begin{solution}[print]
	\begin{align*}
		a^n - b^n = (a - b) \cdot (a^{n - 1} + a^{n - 2} b + \dots + a b^{n - 2} + b^{n - 1})
	\end{align*}
	Therefore, let
	\begin{align*}
		a &= \sqrt[3]{n^3 + 3n}\\
		b &= \sqrt[3]{n^3}
	\end{align*}

	\begin{align*}
		a - b &= \dfrac{a^3 - b^3}{a^2 + a b + b^2}\\
		\therefore \sqrt[3]{n^3 + 3n} - n &= \dfrac{n^3 + 3n - n^3}{(n^3 + 3n)^{\sfrac{2}{3}} + (n^3 + 3n)^{\sfrac{1}{3}} n + n^2}\\
		&= \dfrac{3}{\left( \dfrac{n^3 + 3n}{n^{\sfrac{3}{2}}} \right)^{\sfrac{2}{3}} + \left( \dfrac{n^3 + 3n}{n^3} \right)^{\sfrac{1}{3} n} + n}
	\end{align*}
	Therefore, the limit is 0.
\end{solution}

\begin{question}
	Prove
	\begin{equation*}
		\lim\limits_{n \to \infty} \dfrac{n!}{n^n} = 0
	\end{equation*}
\end{question}

\begin{solution}[print]
	\begin{equation*}
		0 \leq \dfrac{n!}{n^n} = \dfrac{1}{n} \dfrac{2}{n} \dots \dfrac{n}{n} \leq \dfrac{1}{n}\\
	\end{equation*}
	Therefore, by the Sandwich Theorem, $\lim\limits_{n \to \infty} \dfrac{n!}{n^n} = 0$.
\end{solution}

\begin{question}
	Let $a_1 = 3$, $a_{n + 1} = 1 + \sqrt{6 + a_n}$. Prove that $a_n$ converges and find its limit.
\end{question}

\begin{solution}[print]
	If possible, let $\lim\limits_{n \to \infty} a_n = l$.
	\begin{align*}
		a_{n + 1} &= 1 + \sqrt{6 + a_n}\\
		\intertext{Taking the limit on both sides,}
		l &= 1 + \sqrt{6 + l}\\
		\therefore l - 1 &= \sqrt{6 + l}\\
		\therefore l &= \dfrac{3 \pm \sqrt{29}}{2}
	\end{align*}
	As $a_n \geq 0$, $l = \dfrac{3 + \sqrt{29}}{2}$.
	~\\
	\begin{align*}
		a_2 &= 1 + \sqrt{6 + a_1}\\
		&= 1 + \sqrt{6 + 3}\\
		&= 4\\
		\therefore a_2 &> a_1
	\end{align*}
	If possible, let $a_n \geq a_{n - 1}$.\\
	Therefore,
	\begin{align*}
		a_{n + 1} &= 1 + \sqrt{6 + a_n}\\
		&\geq 1 + \sqrt{6 + a_{n + 1}} = a_n
	\end{align*}
	Therefore by induction, $\{a_n\}$ is monotonically increasing.
	~\\
	\begin{align*}
		a_1 &= 3\\
		\therefore a_1 \leq 5
	\end{align*}
	If possible, let $a_n \leq 5$.\\
	Therefore,
	\begin{equation*}
		a_{n + 1} = 1 + \sqrt{6 + a_n} \leq q + \sqrt{11} \leq 5
	\end{equation*}
	Therefore by induction, $\{a_n\}$ is bounded from above by 5.
\end{solution}

\recitation

\subsection{Limit of a Function by Heine}

\begin{definition}
	\begin{equation*}
		\lim\limits_{x \to x_0} f(x) = l
	\end{equation*}
	if for every sequence $x_n$, such that $\lim\limits_{n \to \infty} x_n = x_0$, \begin{equation*}
		\lim\limits_{n \to \infty} f(x_n) = l
	\end{equation*}
\end{definition}

\begin{theorem}
	If $f$ is continuous at $x_0$ and $x_n \to x_0$, then 
	\begin{equation*}
		\lim\limits_{n \to \infty} f(x_n) = f\left( \lim\limits_{n \to \infty} x_n \right) = f_{x_0}
	\end{equation*}
\end{theorem}

\begin{question}
	Calculate $\lim\limits_{n \to \infty} \sqrt[n]{n}$.
\end{question}

\begin{solution}
	Let
	\begin{equation*}
		f(x) = x^{\sfrac{1}{x}}
	\end{equation*}
	Therefore,
	\begin{align*}
		\lim\limits_{x \to \infty} x^{\sfrac{1}{x}} &= \lim\limits_{x \to \infty} e^{\cancelto{0}{\dfrac{\ln x}{x}}}\\
		&= 1
	\end{align*}
\end{solution}

\subsection{Sub-sequences}

\begin{question}
	Find all partial limits and $\overline{\lim}$ and $\underline{\lim}$ of
	\begin{equation*}
		a_n = \left( \cos \dfrac{\pi n}{4} \right)^n
	\end{equation*}
\end{question}

\begin{solution}
	Let $k, z \in \mathbb{Z}$
	\begin{align*}
		\cos \dfrac{\pi n}{4} &= \cos \dfrac{\pi (n + k)}{4}\\
		\therefore \dfrac{\pi n}{4} &= \dfrac{\pi (n + k)}{4} + 2 \pi z\\
		\therefore \pi n &= \pi (n + k) + 8 \pi z\\
		\therefore k &= 8 z
	\end{align*}
	Therefore,
	\begin{align*}
		a_{8k} &= \left( \cos \dfrac{\pi \cdot 8k}{4} \right)^{8k}\\
		&= \left( \cos (2 \pi k) \right)^{8k}\\
		&= 1\\
		a_{8k + 1} &= \left( \cos \dfrac{\pi \cdot (8k + 1)}{4} \right)^{8k + 1}\\
		&= \left( \cos \dfrac{\pi}{4} \right)^{8k + 1}\\
		&= \left( \dfrac{\sqrt{2}}{2} \right)^{8k + 1}\\
		a_{8k + 2} &= \left( \cos \dfrac{\pi \cdot (8k + 2)}{4} \right)^{8k + 2}\\
		&= \left( \cos \dfrac{\pi}{2} \right)^{8k + 2}
	\end{align*}
	Therefore,
	\begin{align*}
		\lim\limits_{k \to \infty} a_{8k} &= 1\\
		\lim\limits_{k \to \infty} a_{8k + 1} &= \lim\limits_{k \to \infty} \left( \dfrac{\sqrt{2}}{2} \right)^{8k + 1}\\
		&= 0
	\end{align*}
	Similarly,
	\begin{align*}
		\lim\limits_{k \to \infty} a_{8k + 2} &= 0\\
		\lim\limits_{k \to \infty} a_{8k + 3} &= 0\\
		\lim\limits_{k \to \infty} a_{8k + 4} &= \lim\limits_{k \to \infty} (-1)^{8k + 4}\\
		&= 1\\
		\lim\limits_{k \to \infty} a_{8k + 5} &= 0\\
		\lim\limits_{k \to \infty} a_{8k + 6} &= 0\\
		\lim\limits_{k \to \infty} a_{8k + 7} &= 0
	\end{align*}
	Therefore, $\{a_n\}$ has two partial limits, $0$ and $1$.
	\begin{align*}
		\overline{\lim} a_n &= 1\\
		\underline{\lim} a_n &= 0
	\end{align*}
\end{solution}

\section{Series}

\begin{definition}[Convergence of a series]
	Let $\{a_n\}$ be a sequence. Let $S_n$ be a sequence of partial sums of $a_n$, s.t.
	\begin{equation*}
		S_n = \sum_{k = 1}^{n} a_k
	\end{equation*}
	The series $\sum_{k = 1}^{\infty} a_k$ is said to converge to $l$ if
	\begin{equation*}
		\lim\limits_{n \to \infty} S_n = l
	\end{equation*}
	that is,
	\begin{equation*}
		\sum_{k = 1}^{\infty} a_k = \lim\limits_{n \to \infty} \sum_{k = 1}^{n} a_k = \lim\limits_{n \to \infty} S_n
	\end{equation*}
\end{definition}

\begin{question}
	Does $\displaystyle \sum_{k = 0}^{\infty} q^k$ where $-1 < q < 1$ converge?
\end{question}

\begin{solution}
	\begin{align*}
		\sum_{k = 0}^{\infty} q^k &= \lim\limits_{n \to \infty} \sum_{k = 0}^{n} q^k\\
		&= \lim\limits_{n \to \infty} \dfrac{1 - q^{n + 1}}{1 - q}\\
		&= \dfrac{1}{1 - q}
	\end{align*}
	Therefore, the series converges.
\end{solution}

\begin{question}
	Does $\displaystyle \sum_{k = 1}^{\infty} \dfrac{1}{k(k + 1)}$ converge?
\end{question}

\begin{solution}
	\begin{align*}
		\sum_{k = 1}^{\infty} \dfrac{1}{k(k + 1)} &= \sum_{k = 1}^{\infty} \left( \dfrac{1}{k} - \dfrac{1}{k + 1} \right)\\
		&= \lim\limits_{n \to \infty} \sum_{k = 1}^{n} \left( \dfrac{1}{k} - \dfrac{1}{k + 1} \right)\\
		&= \lim\limits_{n \to \infty} \left( 1 - \dfrac{1}{n + 1} \right)\\
		&= 1
	\end{align*}
\end{solution}

\begin{question}
	Does $\displaystyle \sum_{k = 1}^{\infty} \left( 1 + \dfrac{1}{k} \right)^k$ converge?
\end{question}

\begin{solution}
	\begin{align*}
		\lim\limits_{k \to \infty} \left( 1 + \dfrac{1}{k} \right)^k &= e\\
		\therefore \lim\limits_{k \to \infty} \left( 1 + \dfrac{1}{k} \right)^k &\neq 0
	\end{align*}
	Therefore, the necessary condition is nt satisfied.
	Hence, the series does not converge.
\end{solution}

\subsection{Comparison Tests for Positive Series}

\begin{theorem}[First Comparison Test]
	If $a_n \ge 0$, $b_n \ge 0$, and $a_n \le b_n$, then
	\begin{enumerate}
		\item If $\sum b_n$ converges, then $\sum a_n$ converges.
		\item If $\sum a_n$ diverges, then $\sum b_n$ diverges.
	\end{enumerate}
	\label{First Comparison Test}
\end{theorem}

\begin{theorem}[Second Comparison Test]
	If $a_n \ge 0$, $b_n \ge 0$ and
	\begin{equation*}
		\lim\limits_{n \to \infty} \dfrac{a_n}{b_n} = l
	\end{equation*}
	where $0 < l < \infty$, then $\sum a_n$ and $\sum b_n$ converge or diverge simultaneously.
	\label{Second Comparison Test}
\end{theorem}

\recitation

\begin{question}
	Suppose the sequence $a_n$ satisfies the condition
	\begin{equation*}
		a_{n + 1} - a_n > \dfrac{1}{n}
	\end{equation*}
	$\forall n \in \mathbb{N}$.\\
	Prove that $\lim\limits_{n \to \infty} a_n = \infty$.
\end{question}

\begin{solution}
	\begin{align*}
		a_{n + 1} &= a_{n + 1} - a_{n} + a_{n} - a_{n - 1} + a_{n - 1} - a_{n - 2} + \dots + a_{2} - a_{1} + a_{1}\\
		&= \sum_{k = 1}^{n} \left( a_{k + 1} - a_{k} \right) + a_{1}\\
		&\ge \sum_{k = 1}^{n} \dfrac{1}{k} + a_1
	\end{align*}
	As the harmonic series diverges, $\sum\limits_{k = 1}^{n} \dfrac{1}{k} + a_1$ diverges.\\
	Therefore, by the \nameref{First Comparison Test}, $\sum\limits_{k = 1}^{\infty} (a_{k + 1} - a_k)$ diverges.
\end{solution}

\begin{question}
	Check the convergence of $\sum\limits_{n = 2}^{\infty} \dfrac{n + \sin n}{n^3 + \cos \pi n}$.
\end{question}

\begin{solution}
	The series is non-negative.
	Therefore, the comparison tests are applicable.
	\begin{align*}
		\dfrac{n + \sin n}{n^3 + \cos \pi n} &\le \dfrac{n + 1}{n^3 - 1}\\
		\therefore \dfrac{n + \sin n}{n^3 + \cos \pi n} &\le \dfrac{2n}{n^3 - \dfrac{n^3}{2}} &\le \dfrac{4}{n^2}
	\end{align*}
	Therefore, by the \nameref{First Comparison Test}, as $\dfrac{4}{n^2}$ converges, $\sum_{n = 2}^{\infty} \dfrac{n + \sin n}{n^3 + \cos \pi n}$ also converges.
\end{solution}

\begin{question}
	Let $a_n \ge 0$ and suppose that $\sum a_n$ converges. Prove that $\sum {a_n}^2$ converges.\\
	Is it true without the assumption $a_n \ge 0$?
\end{question}

\begin{solution}
	As $\sum a_n$ converges, $\lim\limits_{n \to \infty} a_n = 0$.\\
	Therefore, $\exists N \in \mathbb{N}$, such that $\forall n > N$, $a_n < 1$.\\
	Therefore, $\forall n > N$, ${a_n}^2 \le a_n$.
	Hence, as $\sum\limits_{n = N + 1}^{\infty} a_n$ converges, $\sum\limits_{n = N + 1}^{\infty} {a_n}^2$ also converges.
	Hence, $\sum\limits_{n = 1}^{\infty} a_n$ also converges.\\
	~\\
	This is not true without the assumption $a_n \ge 0$, as the argument ${a_n}^2 \le a_n$ does not hold.
\end{solution}

\begin{question}
	For which $\alpha$ does $\sum \left( \sqrt{n + 1} - \sqrt{n} \right)^{\sfrac{\alpha}{2}}$ converge?
\end{question}

\begin{solution}
	\begin{align*}
		\sum \left( \sqrt{n + 1} - \sqrt{n} \right)^{\sfrac{\alpha}{2}} &= \sum \left( \dfrac{n + 1 - n}{\sqrt{n + 1} + \sqrt{n}} \right)^{\sfrac{\alpha}{2}}\\
		&= \sum \left( \dfrac{1}{\sqrt{n + 1} - \sqrt{n}} \right)^{\sfrac{\alpha}{2}}
	\end{align*}
	The series is positive.
	Therefore, the comparison tests are applicable.\\
	Comparing with $\left( \dfrac{1}{\sqrt{n}} \right)^{\sfrac{\alpha}{2}}$,
	\begin{align*}
		\dfrac{\left( \dfrac{1}{\sqrt{n + 1} + \sqrt{n}} \right)^{\sfrac{\alpha}{2}}}{\left( \dfrac{1}{\sqrt{n}} \right)^{\sfrac{\alpha}{2}}} &= \left( \dfrac{\sqrt{n}}{\sqrt{n + 1} + \sqrt{n}} \right)^{\sfrac{\alpha}{2}}\\
		\therefore \lim\limits_{n \to \infty} \left( \dfrac{\sqrt{n}}{\sqrt{n + 1} + \sqrt{n}} \right)^{\sfrac{\alpha}{2}} &= \left( \dfrac{1}{2} \right)^{\sfrac{\alpha}{2}}
	\end{align*}
	$\sum \dfrac{1}{n^{\sfrac{\alpha}{2}}}$ converges if and only if $\dfrac{\alpha}{4} > 1$, i.e. if an inly if $\alpha > 4$.\\
	By the \nameref{Second Comparison Test}, $\sum \dfrac{1}{n^{\sfrac{\alpha}{4}}}$ and the series converge or diverge simultaneously.\\
	Therefore, the series converges for $\alpha > 4$.
\end{solution}

\begin{question}
	Check the convergence of $\sum\limits_{n = 1}^{\infty} \sin \dfrac{1}{n}$.
\end{question}

\begin{solution}
	$\forall n \in \mathbb{N}$, $\sin \dfrac{1}{n} \ge 0$\\
	\begin{align*}
		\lim\limits_{n \to \infty} \dfrac{\sin \dfrac{1}{n}}{\dfrac{1}{n}} &= 1
	\end{align*}
	Therefore, by \nameref{Second Comparison Test}, $\sum \dfrac{1}{n}$ and $\sum \sin \dfrac{1}{n}$ diverge simultaneously.
\end{solution}

\subsection{d'Alembert Criteria (Ratio Test)}

\begin{definition}[Absolute and conditional convergence]
	The series $\sum a_n$ is said to converge absolutely if $\sum |a_n|$ converges.
	The series $\sum a_n$ is said to converge conditionally if it converges but $\sum |a_n|$ diverges.
\end{definition}

\begin{theorem}
	If the series $\sum a_n$ converges absolutely then it converges.
\end{theorem}

\begin{theorem}[d'Alembert Criteria (Ratio Test)]
	\begin{enumerate}
		\item 
			If 
			\begin{equation*}
			\lim\limits_{n \to \infty} \left| \dfrac{a_{n - 1}}{a_n} \right| = L < 1
			\end{equation*}
			then $\sum a_n$ converges absolutely.
		\item 
			If 
			\begin{equation*}
			\lim\limits_{n \to \infty} \left| \dfrac{a_{n - 1}}{a_n} \right| = L > 1
			\end{equation*}
			(including $L = \infty$), then $\sum a_n$ converges diverges.
		\item If $L = 1$, the test does not apply.
	\end{enumerate}
	\label{d'Alembert Criteria (Ratio Test)}
\end{theorem}

\begin{question}
	Check the convergence of $\sum \dfrac{(-1)^n \cdot n^{1000}}{1 \cdot 3 \cdot 5 \cdot \dots \cdot (2n - 1)}$.
\end{question}

\begin{solution}
	\begin{align*}
	\sum_{n = 1}^{\infty} \left| \dfrac{(-1)^n \cdot n^{1000}}{1 \cdot \dots \cdot (2n - 1)} \right| &= \sum_{n = 1}^{\infty} \dfrac{n^{1000}}{1 \cdot \dots \cdot (2n - 1)}
	\end{align*}
	Therefore, by the \nameref{d'Alembert Criteria (Ratio Test)},
	\begin{align*}
		\dfrac{a_{n + 1}}{a_n} &= \dfrac{\dfrac{(n + 1)^{1000}}{1 \cdot \dots \cdot (2n + 1)}}{\dfrac{n^{1000}}{1 \cdot \dots \cdot (2n - 1)}}\\
		&= \left( \dfrac{n + 1}{n} \right)^{1000} \cdot \dfrac{1}{2n + 1}\\
		\therefore \lim\limits_{n to \infty} \left( \dfrac{n + 1}{n} \right)^{1000} \cdot \dfrac{1}{2n + 1} &= 0\\
		\therefore \left( \dfrac{n + 1}{n} \right)^{1000} \cdot \dfrac{1}{2n + 1} &< 1
	\end{align*}
	Therefore, by the \nameref{d'Alembert Criteria (Ratio Test)}, the series converges absolutely, and hence converges.
\end{solution}

\subsection{Cauchy Criteria (Cauchy Root Test)}

\begin{theorem}[Cauchy Criteria (Cauchy Root Test)]
	\begin{enumerate}
		\item
			If 
			\begin{equation*}
			\overline{\lim} \sqrt[n]{|a_n|} = L < 1
			\end{equation*}
			then $\sum a_n$ converges absolutely.
		\item
			If 
			\begin{equation*}
			\overline{\lim} \sqrt[n]{|a_n|} = L > 1
			\end{equation*}
			(including $L = \infty$), then $\sum a_n$ diverges.
		\item If $L = 1$, the test does not apply.
	\end{enumerate}
	\label{Cauchy Criteria (Cauchy Root Test)}
\end{theorem}

\begin{question}
	Check the convergence of $\sum \left( 1 - \dfrac{2}{n} \right)^{n^2}$.
\end{question}

\begin{solution}
	\begin{align*}
		\sqrt[n]{\left( 1 - \dfrac{2}{n} \right)^{n^2}} &= \left( 1 - \dfrac{2}{n} \right)^n\\
		\therefore \lim\limits_{n \to \infty} \left( 1 - \dfrac{2}{n} \right)^n &= e^{-2}\\
		\therefore \lim\limits_{n \to \infty} \left( 1 - \dfrac{2}{n} \right)^n &< 1
	\end{align*}
	Therefore, by the \nameref{Cauchy Criteria (Cauchy Root Test)}, $\sum \left( 1 - \dfrac{2}{n} \right)^{n^2}$ converges.
\end{solution}

\subsection{Leibniz's Criteria}

\begin{definition}[Alternating series]
	The series $\sum\limits_{n = 1}^{\infty} (-1)^{n - 1} a_n$, where all $a_n > 0$ or all $a_n < 0$ is called an alternating series.
\end{definition}

\begin{theorem}[Leibniz's Criteria for Convergence]
	If an alternating series $\sum (-1)^{n - 1} a_n$ with $a_n > 0$ satisfies
	\begin{enumerate}
		\item $a_{n + 1} \le a_n$, i.e. $\{a_n\}$ is monotonically decreasing.
		\item $\lim\limits_{n \to \infty} a_n = 0$
	\end{enumerate}
	then the series $(-1)^{n - 1} a_n$ converges.
	\label{Leibniz's Criteria for Convergence}
\end{theorem}

\begin{question}
	Prove or disprove: There exists $\{a_n\}$, such that $\sum a_n$ converges and $\sum (1 + a_n) a_n$ diverges.
\end{question}

\begin{solution}
	Let $a_n = \dfrac{(-1)^n}{\sqrt{n}}$.\\
	Therefore, by \nameref{Leibniz's Criteria for Convergence}, $\sum \dfrac{(-1)^n}{\sqrt{n}}$ converges.\\
	\begin{align*}
		\sum (1 + a_n) a_n &= \sum \left( 1 + \dfrac{(-1)^n}{\sqrt{n}} \right) \dfrac{(-1)^n}{\sqrt{n}}\\
		&= \sum \left( \dfrac{(-1)^n}{\sqrt{n}} + \dfrac{1}{n} \right)
	\end{align*}
	Therefore, as $\sum \dfrac{1}{n}$ diverges, and $\sum \dfrac{(-1)^n}{\sqrt{n}}$ converges, $\sum \left( \dfrac{1}{n} + \dfrac{(-1)^n}{\sqrt{n}} \right)$ diverges.
\end{solution}

\subsection{Integral Test}

\begin{theorem}[Integral Test]
	If $f(x) : [1, \infty) \to [0, \infty)$ is monotonically decreasing.
	Then, $\sum\limits_{n = 1}^{\infty} f(n)$ and $\int\limits_{1}^{\infty} f(x) \dif x$ converge or diverge simultaneously.
	\label{Integral Test}
\end{theorem}

\begin{question}
	Check the convergence of $\sum\limits_{n = 2}^{\infty} \dfrac{1}{n \ln n}$
\end{question}

\begin{solution}
	Let
	\begin{align*}
		f(x) &= \dfrac{1}{x \ln x}
	\end{align*}
	$f(x)$ is monotonically decreasing.
	Therefore, the \nameref{Integral Test} is applicable.\\
	Therefore,
	\begin{align*}
		\int\limits_{2}^{\infty} \dfrac{1}{x \ln x} \dif x &= \int\limits_{\ln 2}^{\infty} \dfrac{1}{y} \dif y\\
		&= \left. \ln y \right|_{\ln 2}^{\infty}\\
		&= \infty
	\end{align*}
	Therefore, by the integral test, $\sum \dfrac{1}{n \ln n}$ diverges.
\end{solution}

\end{document}
