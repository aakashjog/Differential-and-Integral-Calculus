\documentclass[fleqn, a4paper, 10pt, twoside]{amsart}
\usepackage[top = 2cm, bottom = 2cm, inner = 1cm, outer = 2cm]{geometry}
\usepackage{exsheets}
\usepackage{amsmath, amssymb, amsthm} %standard AMS packages
\usepackage{marginnote} %marginnotes
\usepackage{gensymb} %miscellaneous symbols
\usepackage{commath} %differential symbols
\usepackage{xcolor} %colours
\usepackage{cancel} %cancelling terms
\usepackage{siunitx} %formatting units
\usepackage{tikz, pgfplots} %diagrams
	\usetikzlibrary{calc, hobby, patterns, intersections}
\usepackage{graphicx} %inserting graphics
\usepackage{hyperref} %hyperlinks
\usepackage{datetime} %date and time
\usepackage{ulem} %underline for \emph{}
\usepackage{xfrac, lmodern} %inline fractions
\usepackage{enumerate} %numbered lists
\usepackage{float} %inserting floats
\usepackage{tabularx}
\usepackage{multicol}

\newcommand\numberthis{\addtocounter{equation}{1}\tag{\theequation}} %adds numbers to specific equations in non-numbered list of equations

\newcommand{\AxisRotator}[1][rotate=0]{
	\tikz [x=0.25cm,y=0.60cm,line width=.2ex,-stealth,#1] \draw (0,0) arc (-150:150:1 and 1);%
} %rotation symbols on axes

\theoremstyle{definition}
\newtheorem{example}{Example}
\newtheorem{definition}{Definition}

\theoremstyle{theorem}
\newtheorem{theorem}{Theorem}

\newcommand{\curl}{\mathrm{curl\,}}

\makeatletter
\@addtoreset{section}{part} %resets section numbers in new part
\makeatother

\SetupExSheets{solution/print = true} %prints all solutions by default
%opening
\title{Differential and Integral Calculus : Compendium}
\author{Aakash Jog}
\date{\formatdate{29}{4}{2015}}

\begin{document}

\maketitle
%\setlength{\mathindent}{0pt}

%\tableofcontents

\begin{multicols}{2}

\section{Sequences}

\begin{definition}[Sequences bounded from above]
	$\{a_n\}$ is said to be bounded from above if $\exists M \in \mathbb{R}$, s.t. $a_n \leq M$, $\forall n \in \mathbb{N}$.
	Each such $M$ is called an upper bound of $\{a_n\}$.
\end{definition}

\begin{definition}[Sequences bounded from below]
	$\{a_n\}$ is said to be bounded from below if $\exists m \in \mathbb{R}$, s.t. $a_n \geq M$, $\forall n \in \mathbb{N}$.
	Each such $M$ is called an lower bound of $\{a_n\}$.
\end{definition}

\begin{definition}
	$\{a_n\}$ is said to be bounded if it is bounded from below and bounded from above.
\end{definition}

\begin{definition}[Monotonic increasing sequence]
	A sequence $\{a_n\}$ is called monotonic increasing if $\exists n_0 \in \mathbb{N}$, s.t. $a_n \leq a_{n + 1}$, $\forall n \geq n_0$.
\end{definition}

\begin{definition}[Monotonic decreasing sequence]
	A sequence $\{a_n\}$ is called monotonic decreasing if $\exists n_0 \in \mathbb{N}$, s.t. $a_n \geq a_{n + 1}$, $\forall n \geq n_0$.
\end{definition}

\begin{definition}[Strongly increasing sequence]
	A sequence $\{a_n\}$ is called monotonic increasing if $\exists n_0 \in \mathbb{N}$, s.t. $a_n < a_{n + 1}$, $\forall n \geq n_0$.
\end{definition}

\begin{definition}[Strongly decreasing sequence]
	A sequence $\{a_n\}$ is called monotonic decreasing if $\exists n_0 \in \mathbb{N}$, s.t. $a_n > a_{n + 1}$, $\forall n \geq n_0$.
\end{definition}

\begin{example}
	The sequence $\left\{ \frac{n^2}{2^n} \right\}$ is strongly decreasing.
	However, this is not evident by observing the first few terms.
	$\frac{1}{2}, 1, \frac{9}{8}, 1, \frac{25}{32}, \dots$
	\begin{align*}
		a_n &> a_{n + 1}\\
		\iff \frac{n^2}{2^n} &> \frac{(n + 1)^2}{2^{n + 1}}\\
		\iff 2 n^2 &> (n + 1)^2\\
		\iff \sqrt{2} n &> n + 1\\
		\iff n(\sqrt{2} - 1) &> 1\\
		\iff n &> \frac{1}{\sqrt{2} - 1}\\
		\iff n &> 3
	\end{align*}
\end{example}

\subsection{Limit of a Sequence}

\begin{definition}[Limit of a sequence]
	Let $\{a_n\}$ be a given sequence.
	A number $L$ is said to be the limit of the sequence if $\forall \varepsilon > 0$, $\exists n_0 \in \mathbb{N}$, s.t. $|a_n - L| < \varepsilon$, $\forall n \geq n_0$.
	That is, there are infinitely many terms inside the interval and a finite number of terms outside it.
\end{definition}

\begin{question}
	Prove that 2 is not a limit of $\left\{ \frac{3n + 1}{n} \right\}$.
\end{question}

\begin{solution}
	If possible, let 
	\begin{align*}
		\lim\limits_{n \to \infty} \frac{3n + 1}{n} = 2
	\end{align*}
	Then, $\forall \varepsilon > 0$, $\exists n_0 \in \mathbb{N}$, s.t. $\left| \frac{3n + 1}{n} - 2 \right| < \varepsilon$, $\forall n \geq n_0$.
	However,
	\begin{equation*}
		\left| \frac{3n + 1}{n} - 2 \right| = 1 + \frac{1}{n} > 1
	\end{equation*}
	This is a contradiction for $\varepsilon = \frac{1}{2}$.
	Therefore, 2 is not a limit.
\end{solution}

\begin{theorem}
	If a sequence $\{a_n\}$ has a limit $L$ then the limit is unique.
	\label{uniquness of a limit}
\end{theorem}

\begin{theorem}
	If a sequence $\{a_n\}$ has limit $L$, then the sequence is bounded.
	\label{existence of limit implies boundedness}
\end{theorem}

\begin{theorem}
	Let
	\begin{align*}
		\lim\limits_{n \to \infty} a_n &= a\\
		\lim\limits_{n \to \infty} b_n &= b
	\end{align*}
	and let $c$ be a constant.
	Then,
	\begin{align*}
		\lim c &= c\\
		\lim (c a_n) &= c \lim a_n\\
		\lim (a_n \pm b_n) &= \lim a_n \pm \lim b_n\\
		\lim (a_n b_n) &= \lim a_n \lim b_n\\
		\lim (\frac{a_n}{b_n}) &= \frac{\lim a_n}{\lim b_n} \quad (\textnormal{ if } \lim b \neq 0)
	\end{align*}
	\label{limit arithmetic}
\end{theorem}

\begin{theorem}
	Let $\{b_n\}$ be bounded and let $\lim a_n = 0$. Then,
	\begin{equation*}
		\lim (a_n b_n) = 0
	\end{equation*}
\end{theorem}

\begin{theorem}[Sandwich Theorem]
	Let $\{a_n\}$, $\{b_n\}$, $\{c_n\}$ be three sequences. If
	\begin{equation*}
		\lim a_n = \lim b_n = L
	\end{equation*}
	and $\exists n_0 \in \mathbb{N}$, s.t. $\forall n \geq n_0$, $a_n \leq b_n \leq c_n$.
	Then,
	\begin{equation*}
		\lim b_n = L
	\end{equation*}
	\label{sandwich theorem}
\end{theorem}

\begin{question}
	Calculate $\lim\limits_{n \to \infty} \sqrt[n]{2^n + 3^n}$
\end{question}

\begin{solution}[print]
	\begin{gather*}
		\sqrt[n]{3^n} \leq \sqrt[n]{2^n + 3^n} \leq \sqrt[n]{3^n + 3^n} = \sqrt[3]{2 \cdot 3^n}\\
		\therefore 3 \leq \sqrt[n]{2^n + 3^n} \leq 3 \sqrt[n]{2}
	\end{gather*}
	Therefore, by the \nameref{sandwich theorem}, $\lim\limits_{n \to \infty} \sqrt[n]{2^n + 3^n} = 3$.
\end{solution}

\begin{theorem}
	Any monotonically increasing sequence which is bounded from above converges.
	Similarly, any monotonically decreasing sequence which is bounded from below converges.
	\label{monotonicity and boundedness implies convergence}
\end{theorem}

\begin{question}
	Prove that there exists a limit for $a_n = \underbrace{\sqrt{2 + \sqrt{2 + \sqrt{2 + \dots}}}}_{n \textnormal{ times }}$ and find it.
\end{question}

\begin{solution}[print]
	\begin{equation*}
		a_1 = \sqrt{2} < \sqrt{2 + \sqrt{2}} = a_2
	\end{equation*}
	If possible, let
	\begin{align*}
		a_{n - 1} &< a_n\\
		\therefore \sqrt{2 + a_{n - 1}} &< \sqrt{2 + a_n}\\
		\therefore a_n &< a_{n + 1}
	\end{align*}
	Hence, by induction, $\{a_n\}$ is monotonically increasing.
	~\\
	\begin{equation*}
		a_1 = \sqrt{2} \leq 2
	\end{equation*}
	If possible, let
	\begin{align*}
		a_n &\leq 2\\
		\therefore \sqrt{2 + a_n} &\leq \sqrt{2 + 2}\\
		\therefore a_{n + 1} \leq 2
	\end{align*}
	Hence, by induction, $\{a_n\}$ is bounded from above by 2.
	Therefore, by \nameref{monotonicity and boundedness implies convergence}, $\{a_n\}$ converges.
\end{solution}

\subsection{Sub-sequences}

\begin{definition}[Sub-sequence]
	Let $\{a_n\}_{n = 1} ^{\infty}$ be a sequence.
	Let $\{n_k\}_{k = 1}^{\infty}$ be a strongly increasing sequence of natural numbers.
	Let $\{b_k\}_{k = 1}^{\infty}$ be a sequence such that $b_k = a_{n_k}$.
	Then $\{b_k\}_{k = 1}^{\infty}$ is called a sub-sequence of $\{a_n\}_{n = 1}^{\infty}$.
\end{definition}

\begin{theorem}
	If the sequence $\{a_n\}$ converges to $L$ in a wide sense, i.e. $L$ can be infinite, then any sub-sequence of $\{a_n\}$ converges to the same limit $L$.
	\label{Any subsequence converges to the limit of the sequence}
\end{theorem}

\begin{definition}[Partial limit]
	A real number $a$, which may be infinite, is called a partial limit of the sequence $\{a_n\}$ is there exists a sub-sequence of $\{a_n\}$ which converges to $a$.
\end{definition}

\begin{theorem}[Bolzano-Weierstrass Theorem]
	For any bounded sequence there exists a subsequence which is convergent, s.t. there exists at least one partial limit.
	\label{Bolzano-Weierstrass Theorem}
\end{theorem}

\begin{definition}[Upper partial limit]
	The largest partial limit of a sequence is called the upper partial limit.
	It is denoted by $\overline{\lim} a_n$ or $\limsup a_n$.
\end{definition}

\begin{definition}[Lower partial limit]
	The smallest partial limit of a sequence is called the upper partial limit.
	It is denoted by $\underline{\lim} a_n$ or $\liminf a_n$.
\end{definition}

\begin{theorem}
	If the sequence $\{a_n\}$ is bounded and $\overline{\lim} a_n = \underline{\lim} a_n = a$ then $\exists \lim a_n = a$.
	\label{Equal upper and lower partial limits imply existence of limit}
\end{theorem}

\subsection{Cauchy Characterisation of Convergence}

\begin{definition}
	A sequence $\{a_n\}$ is called a Cauchy sequence if $\forall \varepsilon > 0$, $\exists n_0 \in \mathbb{N}$, s.t. $\forall m, n \geq n_0$, $|a_n - a_m| < \varepsilon$.
\end{definition}

\begin{theorem}[Cauchy Characterisation of Convergence]
	A sequence $\{a_n\}$ converges if and only if it is a Cauchy sequence.
	\label{Cauchy Characterisation of Convergence}
\end{theorem}

\begin{theorem}[Another Formulation of the Cauchy Characterisation Theorem]
	The sequence $\{a_n\}$ converges if and only if $\forall \varepsilon > 0$, $\exists n_0 \in \mathbb{N}$, such that $\forall n \geq n_0$ and $\forall p \in \mathbb{N}$, $|a_{n + p} - a_n| < \varepsilon$.
	\label{Another formulation of the Cauchy characterisation theorem}
\end{theorem}

\end{multicols}

\begin{question}
	Prove that the sequence $a_n = \frac{1}{1^2} + \frac{1}{2^2} + \dots + \frac{1}{n^2}$ is convergent.
\end{question}

\begin{solution}
	\begin{align*}
		|a_{n + p} - a_n| &= \left| \frac{1}{1^2} + \frac{1}{2^2} + \dots + \frac{1}{(n + p)^2} - \left( \frac{1}{1^2} + \frac{1}{2^2} + \dots + \frac{1}{n^2} \right) \right|\\
		&= \frac{1}{(n + 1)^2} + \frac{1}{(n + 2)^2} + \dots + \frac{1}{(n + p)^2}\\
		\therefore |a_{n + p} - a_n| &< \frac{1}{n (n + 1)} + \frac{1}{(n + 1)(n + 2)} + \dots + \frac{1}{(n + p - 1)(n + p)}\\
		\therefore |a_{n + p} - a_n| &< \frac{1}{n} - \cancel{\frac{1}{n + 1}} + \cancel{\frac{1}{n + 1}} + \dots + \cancel{\frac{1}{n + p - 1}} - \frac{1}{n + p}\\
		\therefore |a_{n + p} - a_n| &< \frac{1}{n} - \frac{1}{n + p}\\
		\therefore |a_{n + p} - a_n| &< \frac{1}{n}
	\end{align*}
	Therefore, $\forall \varepsilon > 0$, $\exists n_0 \in \mathbb{N}$, s.t. $\forall n \geq n_0$ and $\forall p \in \mathbb{N}$, $|a_{n + p} - a_n| < \varepsilon$, where $n_0 > \frac{1}{\varepsilon}$.
	\qed
\end{solution}

\begin{question}
	Prove that the sequence
	\begin{equation*}
		a_n = \frac{1}{1} + \frac{1}{n} + \dots + \frac{1}{n}
	\end{equation*}
	diverges.
\end{question}

\begin{solution}
	If possible, let the sequence converge.
	Then, by the \nameref{Cauchy Characterisation of Convergence}, $\forall \varepsilon > 0$, $\exists n_0 \in \mathbb{N}$, s.t. $\forall n \geq n_0$ and $\forall p \in \mathbb{N}$, $|a_{n + p} - a_n| < \varepsilon$.\\
	Therefore,
	\begin{align*}
		|a_{n + p} - a_n| &= \left| \frac{1}{1} + \frac{1}{2} + \dots + \frac{1}{n} + \frac{1}{n + p} - \left( \frac{1}{n} + \dots + \frac{1}{n} \right) \right|\\
		&= \frac{1}{n + 1} + \dots + \frac{1}{n + p}\\
		&\ge p \cdot \frac{1}{n + p}\\
		\therefore |a_{n + p} - a_n| &> \frac{p}{n + p}
	\end{align*}
	If $n = p$,
	\begin{align*}
		\frac{p}{n + p} &= \frac{1}{2}
	\end{align*}
	This contradicts the result obtained from the \nameref{Cauchy Characterisation of Convergence}, for $\varepsilon = \frac{1}{4}$.\\
	Therefore, the sequence diverges.
\end{solution}

\begin{multicols}{2}

\section{Series}

\begin{definition}[$p$-series]
	The series $\sum\limits_{n = 1}^{\infty} \frac{1}{n^p}$ is called the $p$-series.
\end{definition}

\begin{theorem}
	The $p$-series converges for $p > 1$ and diverges for $p \le 1$.
	\label{convergence and divergence of p-series}
\end{theorem}

\begin{theorem}
	If $\sum a_n$ converges, then $\lim\limits_{n \to \infty} a_n = 0$, but the converse is not true.
\end{theorem}

\begin{theorem}
	If $\sum a_n$ and $\sum b_n$ converge, then $\sum (a_n \pm b_n)$ and $\sum c a_n$, where $c$ is a constant, also converge. Also,
	\begin{align*}
		\sum (a_n \pm b_n) &= \sum a_n \pm \sum b_n\\
		\sum (c a_n) &= c \sum a_n
	\end{align*}
\end{theorem}

\subsection{Convergence Criteria}

\subsubsection{Leibniz's Criteria}

\begin{theorem}[Leibniz's Criteria for Convergence]
	If an alternating series $\sum (-1)^{n - 1} a_n$ with $a_n > 0$ satisfies
	\begin{enumerate}
		\item $a_{n + 1} \le a_n$, i.e. $\{a_n\}$ is monotonically decreasing.
		\item $\lim\limits_{n \to \infty} a_n = 0$
	\end{enumerate}
	then the series $(-1)^{n - 1} a_n$ converges.
	\label{Leibniz's Criteria for Convergence}
\end{theorem}

\begin{example}
	The alternating harmonic series $\sum \frac{(-1)^{n - 1}}{n}$ converges as $a_n = \frac{1}{n} > 0$, $a_n$ decreases and $\lim a_n = 0$.
\end{example}

\subsubsection{Comparison Test}

\begin{theorem}[First Comparison Test for Convergence]
	Assume $\exists n_0 \in \mathbb{N}$, such that $a_n \ge 0$, $b_n \ge 0$, $\forall n \ge n_0$.
	\begin{enumerate}
		\item If $a_n \le b_n$, $\forall n \ge n_0$ and $\sum\limits_{n = 1}^{\infty} b_n$ converges, then $\sum\limits_{n = 1}^{\infty} a_n$ converges.
		\item If $a_n \ge b_n$, $\forall n \ge n_0$ and $\sum\limits_{n = 1}^{\infty} b_n$ diverges, then $\sum\limits_{n = 1}^{\infty} a_n$ diverges.
	\end{enumerate}
\end{theorem}

\begin{theorem}[Another Formulation of the Comparison Test for Convergence]
	Assume $\exists n_0 \in \mathbb{N}$, such that $a_n \ge 0$, $b_n \ge 0$, $\forall n \ge n_0$, then if $\lim\limits_{n \to \infty} \frac{a_n}{b_n}$ exists, then $\sum a_n$ and $\sum b_n$ converge or diverge simultaneously.
\end{theorem}

\subsubsection{d'Alembert Criteria (Ratio Test)}

\begin{definition}[Absolute and conditional convergence]
	The series $\sum a_n$ is said to converge absolutely if $\sum |a_n|$ converges.
	The series $\sum a_n$ is said to converge conditionally if it converges but $\sum |a_n|$ diverges.
\end{definition}

\begin{theorem}
	If the series $\sum a_n$ converges absolutely then it converges.
\end{theorem}

\begin{theorem}[d'Alembert Criteria (Ratio Test)]
	\begin{enumerate}
		\item 
			If $\lim\limits_{n \to \infty} \left| \frac{a_{n - 1}}{a_n} \right| = L < 1$ then $\sum a_n$ converges absolutely.
		\item 
			If $\lim\limits_{n \to \infty} \left| \frac{a_{n - 1}}{a_n} \right| = L > 1$(including $L = \infty$), then $\sum a_n$ converges diverges.
		\item
			If $L = 1$, the test does not apply.
	\end{enumerate}
	\label{d'Alembert Criteria (Ratio Test)}
\end{theorem}

\subsubsection{Cauchy Criteria (Cauchy Root Test)}

\begin{theorem}[Cauchy Criteria (Cauchy Root Test)]
	\begin{enumerate}
		\item
			If $\overline{\lim} \sqrt[n]{|a_n|} = L < 1$ then $\sum a_n$ converges absolutely.
		\item
			If $\overline{\lim} \sqrt[n]{|a_n|} = L > 1$ (including $L = \infty$), then $\sum a_n$ diverges.
		\item
			If $L = 1$, the test does not apply.
	\end{enumerate}
	\label{Cauchy Criteria (Cauchy Root Test)}
\end{theorem}

\subsubsection{Integral Test}

\begin{theorem}[Integral Test for Series Convergence]
	Let $f(x)$ be a continuous, non-negative, monotonic decreasing function on $[1,\infty)$ and let $a_n = f(n)$.
	Then the series $\sum\limits_{n = 1}^{\infty} a_n$ converges if and only if the improper integral $\int\limits_{1}^{\infty} f(x) \dif x$ converges.
	\label{Integral Test for Series Convergence}
\end{theorem}

\begin{theorem}
	If the series $\sum a_n$ absolutely converges and the series $\sum b_n$ is obtained from $\sum a_n$ by changing the order of the terms in $\sum a_n$ then $\sum b_n$ also absolutely converges and $\sum b_n = \sum a_n$.
\end{theorem}

\begin{theorem}
	If a series converges then the series with brackets without changing the order of terms also converges.
	That is, if $\sum a_n$ converges, then any series of the form $(a_1 + a_2) + (a_3 + a_4 + a_5) + a_6 + \dots$ also converges.
\end{theorem}

\begin{theorem}
	If a series with brackets converges and the terms in the brackets have the same sign, then the series without brackets also converges.
\end{theorem}

\section{Power Series}

\begin{definition}[Power series]
	The series $\sum\limits_{n = 0}^{\infty} a_n (x - c)^n$ is called a power series.
\end{definition}

\begin{theorem}[Cauchy-Hadamard Theorem]
	For any power series $\sum\limits_{n = 0}^{\infty} a_n (x - c)^n$ there exists the limit, which may be infinity, $R = \frac{1}{\overline{\lim\limits_{n \to \infty}} \sqrt[n]{|a_n|}}$	and the series converges for $|x - c| < R$ and diverges for $|x - c| > R$.
	The end points of the interval, i.e. $x = c - R$ and $x = c + R$ must be separately checked for series convergence.
\end{theorem}

\begin{definition}[Radius of convergence and convergence interval]
	The number $R$ is called the radius of convergence and the interval $|x - c| < R$ is called the convergence interval of the series.
	The point $c$ is called the centre of the convergence interval.
\end{definition}

\begin{theorem}
	If $\exists \lim\limits_{n \to \infty} \left| \frac{a_n}{a_{n + 1}} \right|$, which may be infinite, then, $R = \lim\limits_{n \to \infty} \left| \frac{a_n}{a_{n + 1}} \right|$.
\end{theorem}

\begin{theorem}[Stirling's Approximation]
	For $n \to \infty$, $n! \approx \left( \frac{n}{e} \right)^n \sqrt{2 \pi n}$.
\end{theorem}

\begin{question}
	Find the domain of convergence of $\sum\limits_{n = 1}^{\infty} \dfrac{(2x - 4)^n}{n}$.
\end{question}

\begin{solution}
	\begin{align*}
		\sum_{n = 1}^{\infty} \dfrac{(2x - 4)^n}{n} &= \sum_{n = 1}^{\infty} \dfrac{2^n (x - 2)^n}{n}
	\end{align*}
	Therefore, by Cauchy's Formula for Radius of Convergence,
	\begin{align*}
		R &= \dfrac{1}{\overline{\lim} \sqrt[n]{|a_n|}}\\
		&= \dfrac{1}{\lim\limits_{n \to \infty} \sqrt[n]{\dfrac{2^n}{n}}}\\
		&= \dfrac{1}{\lim\limits_{n \to \infty} \dfrac{2}{\sqrt[n]{n}}}\\
		&= \dfrac{1}{2}
	\end{align*}
	Therefore, the series converges for
	\begin{equation*}
		|x - 2| < \dfrac{1}{2}
	\end{equation*}
	and diverges for
	\begin{equation*}
		|x - 2| > \dfrac{1}{2}
	\end{equation*}
	If $x = \dfrac{5}{2}$, 
	\begin{align*}
		\sum_{n = 1}^{\infty} \dfrac{2^n}{n} \left( \dfrac{5}{2} - 2 \right)^n\\
		&= \sum_{n = 1}^{\infty} \dfrac{1}{n}
	\end{align*}
	Therefore, the series diverges.\\
	If $x = \dfrac{3}{2}$, 
	\begin{align*}
		\sum_{n = 1}^{\infty} \dfrac{2^n}{n} \left( \dfrac{3}{2} - 2 \right)^n\\
		&= \sum_{n = 1}^{\infty} (-1)^n \dfrac{1}{n}
	\end{align*}
	Therefore, by Leibniz's Criteria for Convergence, the series converges.\\
	Therefore, the domain of convergence is $\left[ \dfrac{3}{2} , \dfrac{5}{2} \right)$.
\end{solution}

\begin{question}
	Find the radius of convergence of $\sum\limits_{n = 0}^{\infty} n! x^{n!}$.
\end{question}

\begin{solution}
	\begin{align*}
		\dfrac{1}{\overline{\lim}\sqrt[n]{a_n}} &= x + x + 2 x^2 + 6 x^6 + 24 x^{24} + \dots
	\end{align*}
	Therefore,
	\begin{align*}
		a_n &=
			\begin{cases}
				n &;\quad n = k^2\\
				0 &;\quad n \neq k^2\\
			\end{cases}
	\end{align*}
	Therefore, 
	\begin{align*}
		R &= \dfrac{1}{\lim\limits_{n \to \infty} \sqrt[n]{a_n}}\\
		&= \dfrac{1}{\lim\limits_{k \to \infty} \sqrt[k!]{k!}}\\
		&= 1
	\end{align*}
\end{solution}

\end{multicols}

\end{document}
