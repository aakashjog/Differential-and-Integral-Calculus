\documentclass[fleqn, a4paper, 12pt, twoside]{article}
\usepackage{exsheets}
\usepackage{amsmath, amssymb, amsthm} %standard AMS packages
\usepackage{marginnote} %marginnotes
\usepackage{gensymb} %miscellaneous symbols
\usepackage{commath} %differential symbols
\usepackage{xcolor} %colours
\usepackage{cancel} %cancelling terms
\usepackage{siunitx} %formatting units
\usepackage{tikz, pgfplots} %diagrams
	\usetikzlibrary{calc, hobby, patterns, intersections}
\usepackage{graphicx} %inserting graphics
\usepackage{hyperref} %hyperlinks
\usepackage{datetime} %date and time
\usepackage{ulem} %underline for \emph{}
\usepackage{xfrac} %inline fractions
\usepackage{enumerate} %numbered lists
\usepackage{float} %inserting floats

\newcommand\numberthis{\addtocounter{equation}{1}\tag{\theequation}} %adds numbers to specific equations in non-numbered list of equations

\newcommand{\AxisRotator}[1][rotate=0]{
	\tikz [x=0.25cm,y=0.60cm,line width=.2ex,-stealth,#1] \draw (0,0) arc (-150:150:1 and 1);%
} %rotation symbols on axes

\theoremstyle{definition}
\newtheorem{example}{Example}
\newtheorem{definition}{Definition}

\theoremstyle{theorem}
\newtheorem{theorem}{Theorem}

\newcommand{\curl}{\mathrm{curl\,}}

\makeatletter
\@addtoreset{section}{part} %resets section numbers in new part
\makeatother

%opening
\title{Physics 2}
\author{Aakash Jog}
\date{2014-15}

\begin{document}

\maketitle
%\setlength{\mathindent}{0pt}

\tableofcontents

\newpage
\section{Lecturer Information}

\textbf{Dr. Yakov Yakubov}\\
~\\
Office: Schreiber 233\\
Telephone: {\href{tel:+97236405357}{+972 3-640-5357}\\
E-mail: \href{mailto:yakubov@post.tau.ac.il}{yakubov@post.tau.ac.il}\\

\section{Required Reading}

Protter and Morrey: \textit{A first Course in Real Analysis}, UTM Series, Springer-Verlag, 1991

\section{Additional Reading}

Thomas and Finney, \textit{Calculus and Analytic Geometry}, 9th edition, Addison-Wesley, 1996

\newpage
\part{Sequences and Series}

\section{Sequences}

\begin{definition}[Sequence]
	A sequence of real numbers is a set of numbers which are written in some order. There are infinitely many terms in a sequence. It is denoted by $\{a_n\}_{n = 1}^{\infty}$ or $\{a_n\}$.
\end{definition}

\begin{example}
	$1, \dfrac{1}{2}, \dfrac{1}{3}, \dots$ is called the harmonic sequence.
	\begin{equation*}
		a_n = \dfrac{1}{n}
	\end{equation*}
\end{example}

\begin{example}
	$1, -\dfrac{1}{2}, \dfrac{1}{3}, \dots$ is called the alternating harmonic sequence.
	\begin{equation*}
		a_n = (-1)^{n + 1} \dfrac{1}{n}
	\end{equation*}
\end{example}

\begin{example}
	$\dfrac{1}{2}, \dfrac{2}{3}, \dfrac{3}{4}, \dots$
	\begin{equation*}
		a_n = \dfrac{n}{n + 1}
	\end{equation*}
\end{example}

\begin{example}
	$\dfrac{2}{3}, \dfrac{3}{9}, \dfrac{4}{27}, \dots$
	\begin{align*}
		a_n = \dfrac{n + 1}{3^n}
	\end{align*}
\end{example}

\begin{example}
	The Fibonacci sequence is given by
	\begin{equation*}
		f_n =
			\begin{cases}
				1 &;\quad n = 1, 2\\
				f_{n - 1} + f_{n - 2} &;\quad n \geq 3\\
			\end{cases}
	\end{equation*}
\end{example}

\begin{example}
	A geometric sequence is given by
	\begin{equation*}
		a_n = a_1 q^{n - 1}
	\end{equation*}
	where $q$ is called the common ratio.
\end{example}

\begin{example}
	A geometric sequence is given by
	\begin{equation*}
		a_n = a_1 + d(n - 1)
	\end{equation*}
	where $d$ is called the common difference.
\end{example}

\begin{definition}[Equal sequences]
	Two sequences $\{a_n\}$ and $\{b_n\}$ are said to be equal if $a_n = b_n$, $\forall n \in \mathbb{N}$.
\end{definition}

\begin{definition}[Sequences bounded from above]
	$\{a_n\}$ is said to be bounded from above if $\exists M \in \mathbb{R}$, s.t. $a_n \leq M$, $\forall n \in \mathbb{N}$.
	Each such $M$ is called an upper bound of $\{a_n\}$.
\end{definition}

\begin{definition}[Sequences bounded from below]
	$\{a_n\}$ is said to be bounded from below if $\exists m \in \mathbb{R}$, s.t. $a_n \geq M$, $\forall n \in \mathbb{N}$.
	Each such $M$ is called an lower bound of $\{a_n\}$.
\end{definition}

\begin{definition}
	$\{a_n\}$ is said to be bounded if it is bounded from below and bounded from above.
\end{definition}

\begin{example}
	The sequence $a_n = n^2 + 2$ is not bounded from above but is bounded from below, by all $m \leq 3$.
\end{example}

\begin{example}
	$\left\{ \dfrac{2n - 1}{3n} \right\}$ is bounded.
	\begin{equation*}
		m = 0 \leq \dfrac{2n - 1}{3n} \leq \dfrac{2n}{3n} = \dfrac{2}{3} = M
	\end{equation*}
\end{example}

\begin{definition}[Monotonic increasing sequence]
	A sequence $\{a_n\}$ is called monotonic increasing if $\exists n_0 \in \mathbb{N}$, s.t. $a_n \leq a_{n + 1}$, $\forall n \geq n_0$.
\end{definition}

\begin{definition}[Monotonic decreasing sequence]
	A sequence $\{a_n\}$ is called monotonic decreasing if $\exists n_0 \in \mathbb{N}$, s.t. $a_n \geq a_{n + 1}$, $\forall n \geq n_0$.
\end{definition}

\begin{definition}[Strongly increasing sequence]
	A sequence $\{a_n\}$ is called monotonic increasing if $\exists n_0 \in \mathbb{N}$, s.t. $a_n < a_{n + 1}$, $\forall n \geq n_0$.
\end{definition}

\begin{definition}[Strongly decreasing sequence]
	A sequence $\{a_n\}$ is called monotonic decreasing if $\exists n_0 \in \mathbb{N}$, s.t. $a_n > a_{n + 1}$, $\forall n \geq n_0$.
\end{definition}

\begin{example}
	The sequence $\left\{ \dfrac{n^2}{2^n} \right\}$ is strongly decreasing.
	However, this is not evident by observing the first few terms.
	$\dfrac{1}{2}, 1, \dfrac{9}{8}, 1, \dfrac{25}{32}, \dots$
	\begin{align*}
		a_n &> a_{n + 1}\\
		\iff \dfrac{n^2}{2^n} &> \dfrac{(n + 1)^2}{2^{n + 1}}\\
		\iff 2 n^2 &> (n + 1)^2\\
		\iff \sqrt{2} n &> n + 1\\
		\iff n(\sqrt{2} - 1) &> 1\\
		\iff n &> \dfrac{1}{\sqrt{2} - 1}\\
		\iff n &> 3
	\end{align*}
\end{example}

\begin{question}
	Is $a_n = (-1)^n$ monotonic?
\end{question}

\begin{solution}[print]
	The sequence $-1, 1, -1, 1, \dots$ is not monotonic.
\end{solution}

\subsection{Limit of a Sequence}

\begin{definition}
	Let $\{a_n\}$ be a given sequence.
	A number $L$ is said to be the limit of the sequence if $\forall \varepsilon > 0$, $\exists n_0 \in \mathbb{N}$, s.t. $|a_n - L| < \varepsilon$, $\forall n \geq n_0$.
	That is, there are infinitely many terms inside the interval and a finite number of terms outside it.
\end{definition}

\begin{example}
	The sequence $\{\dfrac{1}{n}\}$ tends to 0, i.e. for any open interval $(-\varepsilon, \varepsilon)$, there are finite number of terms of the sequence outside the interval, and therefore there are infinitely many terms inside the interval.
\end{example}

\begin{question}
	Prove
	\begin{equation*}
		\lim\limits_{n \to \infty} \dfrac{n + 2}{2n - 1} = \dfrac{1}{2}
	\end{equation*}
\end{question}

\begin{solution}
	$\forall \varepsilon > 0$, $\exists n_0 \in \mathbb{N}$
\end{solution}

\begin{question}
	Prove that 2 is not a limit of $\left\{ \dfrac{3n + 1}{n} \right\}$.
\end{question}

\begin{solution}
	If possible, let 
	\begin{align*}
		\lim\limits_{n \to \infty} \dfrac{3n + 1}{n} = 2
	\end{align*}
	Then, $\forall \varepsilon > 0$, $\exists n_0 \in \mathbb{N}$, s.t. $\left| \dfrac{3n + 1}{n} - 2 \right| < \varepsilon$, $\forall n \geq n_0$.
	However,
	\begin{equation*}
		\left| \dfrac{3n + 1}{n} - 2 \right| = 1 + \dfrac{1}{n} > 1
	\end{equation*}
	This is a contradiction for $\varepsilon = \dfrac{1}{2}$.
	Therefore, 2 is not a limit.
\end{solution}

\begin{theorem}
	If a sequence $\{a_n\}$ has a limit $L$ then the limit is unique.
	\label{uniquness of a limit}
\end{theorem}

\begin{proof}
	If possible let there exist two limits $L_1$ and $L_2$.
	Therefore, $\forall \varepsilon > 0$, there exist a finite number of terms in the interval $(L_1 - \varepsilon, L_1 + \varepsilon)$.
	Therefore, there exist a finite number of terms in the interval $(L_2 - \varepsilon, L_2 + \varepsilon)$.
	This contradicts the definition of a limit.
	Therefore, the limit is unique.
\end{proof}

\begin{theorem}
	If a sequence $\{a_n\}$ has limit $L$, then the sequence is bounded.
	\label{existence of limit implies boundedness}
\end{theorem}

\begin{theorem}
	Let
	\begin{align*}
		\lim\limits_{n \to \infty} a_n &= a\\
		\lim\limits_{n \to \infty} b_n &= b
	\end{align*}
	and let $c$ be a constant.
	Then,
	\begin{align*}
		\lim c &= c\\
		\lim (c a_n) &= c \lim a_n\\
		\lim (a_n \pm b_n) &= \lim a_n \pm \lim b_n\\
		\lim (a_n b_n) &= \lim a_n \lim b_n\\
		\lim (\dfrac{a_n}{b_n}) &= \dfrac{\lim a_n}{\lim b_n} \quad (\textnormal{ if } \lim b \neq 0)
	\end{align*}
	\label{limit arithmetic}
\end{theorem}

\begin{theorem}
	Let $\{b_n\}$ be bounded and let $\lim a_n = 0$. Then,
	\begin{equation*}
		\lim (a_n b_n) = 0
	\end{equation*}
\end{theorem}

\begin{theorem}[Sandwich Theorem]
	Let $\{a_n\}$, $\{b_n\}$, $\{c_n\}$ be three sequences. If
	\begin{equation*}
		\lim a_n = \lim b_n = L
	\end{equation*}
	and $\exists n_0 \in \mathbb{N}$, s.t. $\forall n \geq n_0$, $a_n \leq b_n \leq c_n$.
	Then,
	\begin{equation*}
		\lim b_n = L
	\end{equation*}
	\label{sandwich theorem}
\end{theorem}

\begin{question}
	Calculate $\lim\limits_{n \to \infty} \sqrt[n]{2^n + 3^n}$
\end{question}

\begin{solution}[print]
	\begin{gather*}
		\sqrt[n]{3^n} \leq \sqrt[n]{2^n + 3^n} \leq \sqrt[n]{3^n + 3^n} = \sqrt[3]{2 \cdot 3^n}\\
		\therefore 3 \leq \sqrt[n]{2^n + 3^n} \leq 3 \sqrt[n]{2}
	\end{gather*}
	Therefore, by the \nameref{sandwich theorem}, $\lim\limits_{n \to \infty} \sqrt[n]{2^n + 3^n} = 3$.
\end{solution}

\begin{theorem}
	Any monotonically increasing sequence which is bounded from above converges.
	Similarly, any monotonically decreasing sequence which is bounded from below converges.
	\label{monotonicity and boundedness implies convergence}
\end{theorem}

\begin{question}
	Prove that there exists a limit for $a_n = \underbrace{\sqrt{2 + \sqrt{2 + \sqrt{2 + \dots}}}}_{n \textnormal{ times }}$ and find it.
\end{question}

\begin{solution}[print]
	\begin{equation*}
		a_1 = \sqrt{2} < \sqrt{2 + \sqrt{2}} = a_2
	\end{equation*}
	If possible, let
	\begin{align*}
		a_{n - 1} &< a_n\\
		\therefore \sqrt{2 + a_{n - 1}} &< \sqrt{2 + a_n}\\
		\therefore a_n &< a_{n + 1}
	\end{align*}
	Hence, by induction, $\{a_n\}$ is monotonically increasing.
	~\\
	\begin{equation*}
		a_1 = \sqrt{2} \leq 2
	\end{equation*}
	If possible, let
	\begin{align*}
		a_n &\leq 2
		\therefore \sqrt{2 + a_n} &\leq \sqrt{2 + 2}\\
		\therefore a_{n + 1} \leq 2
	\end{align*}
	Hence, by induction, $\{a_n\}$ is bounded from above by 2.
	Therefore, by \nameref{monotonicity and boundedness implies convergence}, $\{a_n\}$ converges.
\end{solution}

\end{document}